\preprint{APS/123-QED, USTC-ICTS/PCFT-24-xx}
\bibliographystyle{apsrev4-1}
\title{
% New Physics in Supernovae at Neutrino Telescopes
% Time Prints of New Physics from Supernovae at Neutrino Telescopes
% Supernova Detection with Neutrino Telescopes as a Portal to New Physics
Supernovae Time Profiles as a Portal to New Physics at Neutrino Telescopes
}

\author{Jeff Lazar}
\email{jlazar@icecube.wisc.edu}
\affiliation{Department of Physics \& Laboratory for Particle Physics and Cosmology, Harvard University, Cambridge, MA 02138, USA}
\affiliation{Department of Physics \& Wisconsin IceCube Particle Astrophysics Center, University of Wisconsin-Madison, Madison, WI 53706, USA}
\author{Ying-Ying Li}
\email{yingyingli@ustc.edu.cn}
\affiliation{Peng Huanwu Center for Fundamental Theory, Hefei, Anhui 230026, China}
\affiliation{Interdisciplinary Center for Theoretical Study, University of Science and Technology of China, Hefei, Anhui 230026, China}
\author{Carlos A. Arg\"{u}elles}
\email{carguelles@g.harvard.edu}
\affiliation{Department of Physics \& Laboratory for Particle Physics and Cosmology, Harvard University, Cambridge, MA 02138, USA}
\author{Vedran Brdar}
\email{vedran.brdar@okstate.edu}
\affiliation{Department of Physics, Oklahoma State University, Stillwater, OK, 74078, USA}

%\date{\today}

\begin{abstract}
Neutrino telescopes, such as IceCube, have promised precise time resolutions at the millisecond level for detecting collective rise signals in all photomultiplier.
We demonstrate that such precise time resolutions will play a crucial role in tracking neutrino signals with novel timing patterns related to weakly interacting massive particles in physics beyond the Standard Model.
Focusing on future Supernova explosion in the operation time of IceCube, Focusing on a future Supernova explosion in the operation time of IceCube as an example, we show that neutrino telescopes can provide unprecedented probes of new particles with both low and high masses, where neutrino signals are advanced and delayed, respectively, while comparable to DUNE and Hyper-K in the intermediate mass region.
\end{abstract}

%\keywords{Suggested keywords}%Use showkeys class option if keyword
                              %display desired
\maketitle