\preprint{APS/123-QED, USTC-ICTS/PCFT-24-xx}

\title{Time Prints of New Physics at Neutrino Telescopes
%Supernova Detection with Neutrino Telescopes as a Portal to New Physics
}

\author{Jeff Lazar}
\email{jlazar@icecube.wisc.edu}
\affiliation{Department of Physics \& Laboratory for Particle Physics and Cosmology, Harvard University, Cambridge, MA 02138, USA}
\affiliation{Department of Physics \& Wisconsin IceCube Particle Astrophysics Center, University of Wisconsin-Madison, Madison, WI 53706, USA}
\author{Ying-Ying Li}
\email{yingyingli@ustc.edu.cn}
\affiliation{Peng Huanwu Center for Fundamental Theory, Hefei, Anhui 230026, China}
\affiliation{Interdisciplinary Center for Theoretical Study, University of Science and Technology of China, Hefei, Anhui 230026, China}
\author{Carlos A. Arg\"{u}elles}
\email{carguelles@g.harvard.edu}
\affiliation{Department of Physics \& Laboratory for Particle Physics and Cosmology, Harvard University, Cambridge, MA 02138, USA}
\author{Vedran Brdar}
\email{vedran.brdar@cern.ch}
\affiliation{Department of Physics, Oklahoma State University, Stillwater, OK, 74078, USA}
\affiliation{Theoretical Physics Department, CERN, Esplande des Particules, 1211 Geneva 23, Switzerland}

\date{\today}

\begin{abstract}
More than 30 years ago, a handful of neutrino events from SN 1987A was recorded in several 
neutrino experiments. In this decade, several large-scale neutrino experiments will operate and be able to detect thousands of neutrino events from the next galactic supernova (SN). Such detection would not only lead to a great progress in our understanding of the astrophysical properties of SNe, but would also have a great importance in testing light dark sector models. Traditionally, the new physics constraints from SNe have been associated with SN energy loss arguments. Recently, it was shown that if light new physics states streaming out of the SN decay to neutrinos, non-observation of additional events at present and near future neutrino experiments would lead to limits that surpass the energy loss ones. In this work, focusing on the two representative new physics scenarios, we perform sensitivity projections for new physics searches at several neutrino telescopes, in particular IceCube, where a galactic SN event would be observed through the sharp rise in  photomultiplier counts. Based on these counts, new physics induced events can be constrained. The derived sensitivities strongly exceed constraints from both energy loss and non-observation of additional neutrino events from SN 1987A. We also release a code that allows for efficient computation of sensitivities for new physics searches at neutrino telescopes in the context of galactic SN event. An arbitrary light dark sector model can be studied and the only requirement is that the decay of a light dark sector particle(s) to neutrinos is realized.      
\end{abstract}

%\keywords{Suggested keywords}%Use showkeys class option if keyword
                              %display desired
\maketitle