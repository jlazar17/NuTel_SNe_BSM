\textbf{\textit{Results}}---
\begin{figure}[t!]
    \centering
    \includegraphics[width=0.47\textwidth]{figures/majoran_sensitivity}
    \caption{\textbf{\textit{Exclusion sensitivities for the Majoron case.}}
    The lines on this plot show the exclusion sensitivity for IceCube, and two next-generation neutrino experiments, Hyper-Kamiokande and DUNE.
    Additionally, we also show the reaches by SN cooling (blue) and by the lack of observation of high-energy neutrinos from SN1987A (grey) \cite{Fiorillo:2022cdq}, with the width of the shading region quantifying uncertainties from SN modeling.
    }
    \label{fig:sensitivity}
\end{figure}
Looking for rapid raises within various time windows at neutrino telescope in the occurrence of future nearby SN explosion will provide exceptional tests of the parameter space. Assuming the SN event happens in the galaxy at a distance $D_{\rm SN}=\unit[10]{kpc}$, which is not unlikely~\cite{Reed:2005en,Rozwadowska:2020nab}, we consider the currently operating IceCube neutrino observatory as an example. 
The expected exclusion limit are shown in \cref{fig:sensitivity} (dark red) if no excess of hits on top of those from background noise and the standard neutrino flux were observed. DUNE and Hyper-K will be operating and be able to test the parameter of Majorons by looking for time-integrated high energy neutrinos following \cite{Fiorillo:2022cdq} if such SN event happens during their operation time. We present the estimated reaches of DUNE and Hyper-K in \cref{fig:sensitivity} as red and orange lines, respectively, leaving aside careful detector analysis, together with current limits \cite{Fiorillo:2022cdq} from the aforementioned energy loss requirement (blue band), and non-observation of high energy neutrinos from SN 1987A (gray band) in \cref{fig:sensitivity}. 

We observe that while for the intermediate mass region, the constraints from IceCube are comparable to that from DUNE and Hyper-K, at both low mass region $m_\phi \lesssim \unit[10]{MeV}$ and high mass region $m_\phi \gtrsim \unit[200]{MeV}$, IceCube can provide stronger constraints. This is the outcome we expect. For light $\phi$ case with negligible time delay, the neutrino signals can  arrive at the detector even before the peak of standard neutrino flux arrives. The resulting hits would appear in a time window separated from that when most of standard flux contribution appears, as we show in the left panel of \cref{fig:hits_and_likelihood}, thus enhancing the reaches. On the high mass end, neutrino signals will arrive much later than the standard case, where a time window at late time can be considered to reduce the standard neutrino contaminations. 

Uncertainties on the analysis from different modeling SN explosions are minor, as studies in \cite{li2023old} and potential discrepancies between data and modeling are unresolved questions.
Nevertheless, we point out that the discrepancy could be within $2\sigma$ level which will only change our exclusion limit by a small factor.
We also provide the code used in this work so that the impact of SN modeling uncertainties on the sensitivity of neutrino telescopes may be evaluated in the future.

\textbf{\textit{Outlook}}---
We demonstrated that neutrino telescope such as IceCube provides unprecedented probes of new physics, such as Majoron neutrinos and sterile neutrinos with magnetic portal, via measuring Supernova time profiles to a window as short as $\mathcal{O}(0.01)$ sec. Signatures outside the standard signal time window, including early-glow and after-glow, can be probed at neutrino telescope which enhances probes of new physics.
This analysis can also be extended to probing new particles such as sterile neutrinos with mixing portal where the resulting time profile requires more dedicated studies, especially for resonantly conversions. 

\begin{acknowledgments}
We would like to thank xxx for stimulating discussions about this work. Y.-Y. L is supported by the NSF of China through Grant
No. 12247103 and Grant No. xxx.
\end{acknowledgments}
