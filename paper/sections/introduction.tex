
\textbf{\textit{Introduction}}---
The Standard Model (SM) is a remarkable but incomplete theory. Puzzles such as the non-vanishing neutrino mass, the origin of observed matter-antimatter asymmetry, and the nature of dark matter, among others, seek explanations beyond the Standard Model (BSM). There are also no firmly established clues on the energy scale at which the new particles should appear. To this end, numerous BSM searches across the scales have been performed, from collider searches at high energies \cite{Nath:2010zj} to studies of cosmic microwave background at temperatures near absolute zero \cite{Baumann:2015rya}. Core collapse supernovae (CCSNe) release most energy in neutrinos, offering unique opportunities to test BSM physics via interactions with neutrinos.
The duration of the burst~\cite{Kamiokande-II:1987idp,Bionta:1987qt,Baksan} and inferred total energy of neutrinos~\cite{Loredo:2001rx,Pagliaroli:2008ur,Huedepohl2010} from SN 1987A---the only CCSN observed in neutrinos---have already given tentative constraints on the energies that could be taken away by any light BSM particle copiously produced in the interior of a star.
Indeed, many different BSM scenarios models have been constrained in this way, see \textit{e.g.}~\cite{Raffelt:2011nc,Arguelles:2016uwb,Suliga:2020vpz,Lucente:2021hbp,Caputo:2022rca,Caputo:2021rux,PhysRevD.100.083002,DeRocco:2019njg,Kazanas:2014mca,Magill:2018jla}.


If, additionally, BSM particles emitted from supernovae can decay to neutrinos en route to Earth, even stronger limits can be set, as shown for instance in \cite{Fiorillo:2022cdq} for Majoron-like bosons and in \cite{Brdar:2023tmi} for a realization featuring neutrino magnetic moment portal. Specifically, unlike neutrinos inside the supernova core, produced BSM states can stream out without further interactions. Consequently, they exit with higher energies than neutrinos, typically around $\mathcal{O}(100)$ MeV. Provided such particles decay to neutrinos, strong limits can be set from the fact that $\mathcal{O}(100)$ MeV energy neutrinos were not recorded from SN 1987A \cite{Fiorillo:2022cdq, Brdar:2023tmi}. So far, such strategies were utilized for neutrino experiments operating during the SN 1987A event as well as future DUNE \cite{DUNE:2015lol} and Hyper-Kamiokande \cite{Hyper-Kamiokande:2018ofw} in light of anticipated galactic supernova event. 
%However, as being decoupled immediately, it is possible that new states can exit the supernova at an earlier time instead---an effect that has not been previously considered.
%This early-glow signal can also be induced by light new states produced from $\nu_e$ scattering inside the supernova given that $\nu_e$ burst happens at an earlier time than the other neutrino flavors \cite{Brdar_2018}. 

On the other hand, neutrino telescopes such as IceCube has pioneered in detecting neutrinos at $\mathcal{O}$(PeV) energies \cite{IceCube:2014stg,IceCube:2018cha} and has identified several specific astrophysical neutrino sources \cite{IceCube:2018cha,IceCube:2023ame,IceCube:2022der}.
IceCube is also expected to be able to detect next galactic supernova  \cite{Kopke_2011}. In fact, search for MeV neutrinos from optically obscured galactic supernovae was recently performed \cite{IceCube:2023ogt} as well as the search for temporal correlation of MeV neutrino events with fast radio bursts \cite{IceCube:2019acm}. In both cases, the observable is a collective rise in all photomultiplier rates on top of the background noise  within a certain time window and it turns out that IceCube is sensitive to intervals as short as $\mathcal{O}(0.01)$ s. 
In this work, we demonstrate that such precise time resolution will play a crucial role in testing new physics from supernovae.   

 Once produced, the new state can be non-relativistic and travel slowly, causing time delays of the secondary neutrinos. However, it is also possible that new states are rather light implying that they may even exit the supernova at earlier times with respect to certain flavors of SM neutrinos. Such an early BSM signal, which to the best of our knowledge has not been previously considered, can for instance be induced by light BSM states produced from $\nu_e$ scattering inside the supernova; then subsequently decaying to $\bar{\nu}_e$ which chiefly induces the signal at IceCube. This would happen at early stages of supernova, around the neutralization burst, when $\nu_e$ are present and antineutrinos did not get produced yet. Such novel timing patterns generically exist in models with new weakly interacting particles, such as the aforementioned Majoron model \cite{Fiorillo:2022cdq} and neutrino magnetic moment portal \cite{Magill:2018jla,Brdar:2020quo}.
%, but also in scenarios with the mixing portals \cite{Arg_elles_2019,Suliga_2020,Chauhan:2023sci}.
Via the timing measurement, we will show that neutrino telescopes will be able to provide powerful constraints on the BSM parameter space in association to forthcoming galactic supernova event.\\