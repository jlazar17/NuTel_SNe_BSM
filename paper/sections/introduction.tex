\textbf{\textit{Introduction}}---
The Standard Model (SM) is a remarkable but incomplete theory. Puzzles such as the non-vanishing tiny neutrino mass, the origin of observed matter-antimatter asymmetry, and the nature of dark matter, among others, all seek explanations beyond the Standard Model (BSM). However, there are no affirmative clues for the energy scale the yet undiscovered particles should appear. Numerous BSM searches across the scales have been performed, from collider searches at high energies \cite{Nath:2010zj} to studies of cosmic microwave background at temperatures near absolute zero \cite{Baumann:2015rya}. Galactic SN explosion, where majority of emitted energy is released in the form of neutrinos, offers unique opportunities in testing BSM physics via interactions with active neutrinos. So far we have observed SN 1987A in neutrinos. The duration of such a signal \cite{Kamiokande-II:1987idp,Bionta:1987qt,Baksan} as well as the inferred total emitted energy \cite{Loredo:2001rx,Pagliaroli:2008ur,Huedepohl2010} allow us already to set a tentative constraint on the energies that could be taken away from any light BSM particle copiously produced in the interior of a star. Indeed, many different BSM scenarios models have been constrained in this way, see \emph{e.g.} \cite{Raffelt:2011nc,Arguelles:2016uwb,Suliga:2020vpz,Lucente:2021hbp,Caputo:2022rca,Caputo:2021rux,PhysRevD.100.083002,DeRocco:2019njg,Kazanas:2014mca,Magill:2018jla}.

If, additionally, new physics particles, emitted from the SN core, can decay to neutrinos, even stronger limits can be set. In particular, for the standard case, neutrinos interact frequently in the core and eventually leave the star with $\mathcal{O}(10)$ MeV energy. New states after being produced can stream out without further interactions. Consequently, they typically exit with higher energies, subsequently decaying into neutrinos with an energy on the order of $\mathcal{O}(100)$ MeV. Limits can be set from the fact that such high energy neutrinos were not recorded from SN 1987A \cite{Kamiokande-II:1987idp,Bionta:1987qt,Baksan, Fiorillo:2022cdq, Brdar:2023tmi}. Once produced from SN, the new state can be non-relativistic and travels slowly, causing time delays to the arrival time of its daughter particle which was also recently employed \cite{Brdar:2023tmi} for a time window around one day. However, as being decoupled immediately, it is possible that new states can exit the supernova at an earlier time instead—an effect that has not been previously considered. These two effects in combination would cause advanced or delayed neutrino signals to the standard case.

On the other hand, neutrino telescopes such as IceCube have not only pioneered in detecting neutrinos at ultra-high energies \cite{IceCube:2014stg,IceCube:2018cha} and from specific extragalactic sources \cite{IceCube:2023ame,IceCube:2022der}, but also in testing temporal correlation of MeV neutrino events with fast radio bursts with its precise timing measurements \cite{IceCube:2019acm}. In this work, we demonstrate that such timing measurements will play a crucial role in tracking the delayed neutrino signals, and interestingly reveil new features of advanced neutrino signals coming from the immediate-decoupling of its mother particles. This novel timing patterns generically exist in models with new weakly interacting particles, such as sterile neutrino via the dipole portal \cite{Magill:2018jla,Brdar:2020quo,Brdar:2023tmi} or mixing portals \cite{Arg_elles_2019, Suliga_2020}, Majoron models \cite{Fiorillo:2022cdq}. Once produced inside the SN, these particles stream out at an earlier time than the standard neutrino signal. Meanwhile, for non-negligible mass, these particle would travel slower than active neutrino before decaying or converting back to active neutrinos. Via the timing measurement, we will show that neutrino telescope can provide unprecedented constraints on the parameter space in BSM.\\ 