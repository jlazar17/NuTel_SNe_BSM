\textbf{\textit{Introduction}}---
Standard Model (SM) is a remarkable but also an incomplete theory. The discovery of neutrino oscillations implying non-vanishing neutrino mass, the existence of dark matter and the baryon asymmetry of the Universe seek a Beyond the Standard Model (BSM) explanation. However, we presently do not understand at what energy scale the yet undiscovered particles should appear. Numerous BSM searches across the scales have been performed, from collider searches at high energies \cite{Nath:2010zj} to studies of cosmic microwave background at temperatures near absolute zero \cite{Baumann:2015rya}. In this work we will focus on testing BSM using anticipated galactic SN explosion, where majority of emitted energy is released in the form of neutrinos carrying $\mathcal{O}(10)$ MeV energy. So far we have observed SN 1987A in neutrinos and the duration of such a signal \cite{Kamiokande-II:1987idp,Bionta:1987qt,Baksan} as well as the inferred total emitted energy \cite{Loredo:2001rx,Pagliaroli:2008ur,Huedepohl2010} allow us already to set a tentative constraint that any light BSM particle copiously produced in the interior of a star could not have taken more energy than what was released in neutrinos. Indeed, many different BSM scenarios models have been constrained in this way, see \emph{e.g.} \cite{Raffelt:2011nc,Arguelles:2016uwb,Suliga:2020vpz,Lucente:2021hbp,Caputo:2022rca,Caputo:2021rux,PhysRevD.100.083002,DeRocco:2019njg,Kazanas:2014mca,Magill:2018jla}.

 If, additionally, new physics particles, emitted from the SN core, can decay to neutrinos, even stronger limits can be set. In particular, after being produced, such light states stream out without further interactions and therefore typically exit with the energy of $\mathcal{O}(100)$ MeV. Neutrinos, on the other hand, interact frequently in the core and eventually leave the star with $\mathcal{O}(10)$ MeV energy. If $\mathcal{O}(100)$ MeV state can decay to neutrinos, limits can be set from the fact that $\mathcal{O}(100)$ MeV neutrinos were not recorded from SN 1987A \cite{Kamiokande-II:1987idp,Bionta:1987qt,Baksan}. once produced from SN, the new state can be non-relativistic and travels slowly, high energy neutrinos from such state decay can arrive at the detector at later time. 
 Combined with the time delay signal, searching for higher energy neutrinos from decay was recently employed to constrain neutrinophilic bosons \cite{Fiorillo:2022cdq} (throughout this work we will refer to this scenario as the Majoron model) and transition magnetic moment between active and sterile neutrino \cite{Brdar:2023tmi}. %\yy{shall we, from here, start to talk about the bright potential of IceCube?}
 We will employ these two models, which contain light BSM boson and fermion, respectively,  as a case study to illustrate that there is a plethora of low scale neutrinophilic models that can be tested.

In this work we will be focused on demonstrating the exquisite power of neutrino telescopes for testing light BSM physics from SNe. In doing so, we will chiefly focus on IceCube which has pioneered in detecting neutrinos at high energies \cite{IceCube:2014stg,IceCube:2018cha} and also from specific extragalactic sources \cite{IceCube:2023ame,IceCube:2022der}. IceCube is not optimized for SN neutrino detection at MeV energies and yet it will be very successful; IceCube would discover SN through the increase in detector noise during the SN observational window \cite{Kopke:2017req,SN2011}. In addition to discovering SN via neutrinos, in this work we will demonstrate that IceCube can constrain or discover new physics particles that are emitted from SN and decaying to neutrinos en route to Earth. Such neutrinos would induce additional activity in the photomultipliers and based on that, we will present sensitivity projections which are stronger than the present limits. While we will be focused on the two specific models mentioned above, together with this paper we also release 
a code that allows for efficient and robust calculation of projected sensitivity to an arbitrary neutrinophilic new physics realization.

